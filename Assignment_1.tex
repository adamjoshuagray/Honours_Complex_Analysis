\documentclass{unswmaths}
\usepackage{unswshortcuts}
\usepackage{dsfont}
\begin{document}
\author{Adam J. Gray}
\title{Assignment 1}
\subject{Complex Analysis}
\studentno{3329798}

\newcommand{\llra}{\Leftrightarrow}

\unswtitle

\section*{Question 5}
We proprose the function
\begin{align*}
    f(z) := \frac{z}{\log(|z|)}
\end{align*}
and prove that it satisfies the desired attributes.
If we write $ z = re^{i\theta} $ and $ f(z) = u(r, \theta) + iv(r, \theta) $ then we can use the polar form of the Cauchy Riemann equations.

It is not hard to show that that Cauchy Riemann equations have an equivelent polar form
\begin{align*}
    \frac{\partial u}{\partial r} = \frac{1}{r} \frac{\partial v}{\partial \theta} \ \ \ \ \ \  \frac{\partial v}{\partial r} = - \frac{1}{r} \frac{\partial u}{\partial \theta}
\end{align*}

then see that
\begin{align*}
    \frac{\partial u}{\partial r} &= \frac{\partial}{\partial r} \left[ \frac{\cos(\theta)}{\log(r)}\right] \\
        &= \frac{-\cos(\theta)}{r \log^2(r)} \\
    \text{ and } \\
    \frac{\partial v}{\partial \theta} &= \frac{\partial}{\partial \theta} \left[ \frac{\sin(\theta)}{\log(r)}\right] \\
        &= \frac{\cos(\theta)}{\log(r)} 
\end{align*}
\end{document}
