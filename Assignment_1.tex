\documentclass{unswmaths}
\usepackage{unswshortcuts}
\usepackage{dsfont}
\begin{document}
\author{Adam J. Gray}
\title{Assignment 1}
\subject{Complex Analysis}
\studentno{3329798}

\newcommand{\llra}{\Leftrightarrow}

\unswtitle

\section{}

Write $ s = \sigma + it $.  Then consider
\begin{align*}
    \zeta(s) - \frac{1}{s-1} &= \sum_{n=1}^\infty n^{-s} - \frac{1}{s-1} \\
        &= \sum_{n=1}^\infty \left[ n^{-s} \int_{n}^{n+1} x^{-s} dx \right] \\
        &= \sum_{n=1}^\infty \int_n^{n+1} (n^{-s} - x^{-s})dx.
\end{align*}
Now note that 
\begin{align*}
    |n^{-s} - x^{-s}| = \left| s \int_{n}^{x} y^{-1-s}dy \right| \leq |s|n^{-1-\sigma}
\end{align*}
for $ x \in [n, n+1] $
and hence
\begin{align*}
   \int_n^{n+1} (n^{-s} - x^{-s}) dx \leq |s|n^{-1-\sigma}  
\end{align*}
which means that
\begin{align*}
     \sum_{n=1}^\infty \int_n^{n+1} (n^{-s} - x^{-s})dx
\end{align*}
converges absolutely on compact subsets of $ \Re(s) > 0 $.
Now as each term in the sum is an analytic function then the sum is an analytic function for $ \sigma > 0 $. 
So
\begin{align*}
    \zeta(s) = \sum_{n=1}^\infty \int_{n}^{n+1}(n^{-s} - x^{-s})dx + \frac{1}{s-1} 
\end{align*}
defines an analytic continuation of $ \zeta(s) $ for $ \Re(s) > 0 $ and $ s \neq 1 $.
\section{}
Since $ f_1 $ is complex differentiable on $ \Cplx $, $ f_1 $ cannot have any singularities in the unit ball and so all the of singularities of $f_1 / f_2 $ in the unit ball come from $ 1/ f_2 $. 

We therefore turn our attention to $ 1 / f_2 $. It is simple to see that if $ 1/f_2 $ has infinitely many poles in the unit ball then $ f_2 $ has infinitely many zeros in the unit ball (at the same locations). Say these zeros are at $ \{ z_n \}_n $ then by the complex analogue of the Bolzano-Wierstrass theorem there exists a subsequence $ \{ x_{n_k} \}_k \subseteq \{ x_n \}_n $ such that $ \lim_{k\lra\infty} x_{n_k} $ exists. 

Suppose $ x^* $ is the limit point of this sequence then by theorem 1 of lecture notes 4, we have that $ f_2 \equiv 0 $ for all $ z $ in the unit ball. 

This means that if $ f_1 / f_2 $ has infinitely many poles in the unit ball then $ f_2 \equiv 0 $ in the unit ball. So if we disregard this degenerate case then $ f_1 / f_2 $ cannot have infinitely many poles in the unit ball.

\section{}
\begin{align*}
    w(z) := \sqrt{k(\rho)} \operatorname{sn}\left( \frac{2K}{\pi} \sin^{-1} z ;\rho \right)
\end{align*}
where
\begin{align*}
    \rho = \left( \frac{a-b}{a+b}\right)^2
\end{align*}
and where $ sn $ corresponds the the inversion of the Jacobi Elliptic function of the first kind,
is the conformal mapping of the ellipse with foci $ \pm \sqrt{k(\rho)} $. A full derivation of this mapping, can be found in \emph{Conformal Mapping} by Zeev Nehari.
\section{}

To do this we just have to prove equivelence of the norm to the $\sup $ norm.

\begin{align*}
    \int \int_{B_{R}(z)} f dA = \pi R^2 f(z) 
\end{align*} 
where $ B_{R}(z) $ is a unit ball centered at $ z $ of radius $ R $ because $ f = u + iv $ has $ u $ and $ v $ harmonic.
Then
\begin{align*}
    |f(z)| \leq \frac{1}{\pi R^2}\int \int_{B_{R}(z)} |f(z)| dA \leq \frac{1}{\pi R^2} \sqrt{\int \int_{B_{R}(z)} |f(z)|^2 dA}
\end{align*}
Thus 
\begin{align*}
    \sup_{|z| \leq R} |f(z)| \leq \frac{1}{\pi R^2} \sqrt{\int \int_{B_{R}(0)}|f(z)|dA}.
\end{align*}
Next see that by the integral estimation lemma we have the opposite
\begin{align*}
     \sup_{|z| \leq R} |f(z)| \geq \frac{1}{\pi R^2} \sqrt{\int \int_{B_{R}(0)}|f(z)|dA}   
\end{align*}
and hence the norms are equivelent. Note that by setting $ R = 1 $ we have a result specific to the question.
\section{}
\begin{align*}
    f(z) := \sum_{n=1}^\infty z^{2^n}
\end{align*}
Firstly we need to show that $ f(z) $ is complex differentiable inside the unit ball.

To do that note firstly
\begin{align*}
    \sum_{n=1}^\infty \left| z^{2^n} \right| \leq \sum_{n=1}^\infty |z^n| 
\end{align*}
so long as $ |z| < 1 $, and so by the comparison test $ f(z) $ must converge inside the unit ball.

Next note that $ z^{2^n} $ is clearly complex differentiable because $ z $ is complex differentiable. 
So in the unit ball $ f(z) $ is the sum of complex differentiable functions, and the sum converges absolutely. Therefore, $ f(z) $ is complex differentaible inside the unit ball.

Let us next show that for $ z = e^{\frac{k}{2^j} \pi} $, $f(z) $ does not converge. 
See that
\begin{align*}
    f(e^{\frac{k}{2^j}\pi}) &= \sum_{n=1}^\infty e^{2^{n-j} k\pi} \\
        &= \sum_{n=1}^{n=j} e^{2^(n-j)k\pi} + \sum_{n=1}^\infty 1
\end{align*}
and so $ f(e\frac{k\pi}{2^j}) $ does not converge. It is well known that numbers of the form $ \mathbb{Q}_{\pi} := e\frac{k\pi}{2^j} $ are dense on the boundry of the unit ball. 

So for some $ z^* $ such that $ |z^*| = 1 $
\begin{align*}
    \lim_{z\lra z^*} f(z) = \underbrace{\lim_{z \lra z^*, z \in \mathbb{Q}_{\pi}} f(z)}_{\text{does not exist}}
\end{align*}
and hence $ \lim_{z \lra z^*}f(z) $ does not exist. 



\end{document}
