\documentclass{unswmaths}
\usepackage{unswshortcuts}
\usepackage{dsfont}
\begin{document}
\author{Adam J. Gray}
\title{Assignment 1}
\subject{Complex Analysis}
\studentno{3329798}

\newcommand{\llra}{\Leftrightarrow}

\unswtitle

\section{}
\section{}
Since $ f_1 $ is complex differentiable on $ \Cplx $, $ f_1 $ cannot have any singularities in the unit ball and so all the of singularities of $f_1 / f_2 $ in the unit ball come from $ 1/ f_2 $. 

We therefore turn our attention to $ 1 / f_2 $. It is simple to see that if $ 1/f_2 $ has infinitely many poles in the unit ball then $ f_2 $ has infinitely many zeros in the unit ball (at the same locations). Say these zeros are at $ \{ z_n \}_n $ then by the complex analogue of the Bolzano-Wierstrass theorem there exists a subsequence $ \{ x_{n_k} \}_k \subseteq \{ x_n \}_n $ such that $ \lim_{k\lra\infty} x_{n_k} $ exists. 

Suppose $ x^* $ is the limit point of this sequence then by theorem 1 of lecture notes 4, we have that $ f_2 \equiv 0 $ for all $ z $ in the unit ball. 

This means that if $ f_1 / f_2 $ has infinitely many poles in the unit ball then $ f_2 \equiv 0 $ in the unit ball. So if we disregard this degenerate case then $ f_1 / f_2 $ cannot have infinitely many poles in the unit ball.

\section{}

\end{document}
