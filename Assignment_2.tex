\documentclass{unswmaths}
\usepackage{unswshortcuts}
\usepackage{dsfont}
\begin{document}
\author{Adam J. Gray}
\title{Assignment 1}
\subject{Complex Analysis}
\studentno{3329798}

\newcommand{\llra}{\Leftrightarrow}

\unswtitle

\section{}
\section{}
\section{}

Let $ G_1 $ be the group of all conformal mappings from the unit ball into itself and let $ G_2 $ be the set of all conformal mappings from the upper half plane into itself. The group operation in both cases is function composition.  We claim that these two groups are isomorphic. 

Let $ D_1 = \{ z \in \Cplx : |z| < 1 \} $ and let $ U = \{ z : \Im(z) > 1 \} $.

Define the mapping $ M : D_1 \lra U $ by
$$ M(z) = \frac{(z+1)(1-i)}{2(z-i)}.$$
This is a Mobius transformation, so it is a conformal mapping and importantly bijective. 

We can produce an \emph{induced} transformation $ M_G : G_1 \lra G_2 $ where for $ f \in G_1 $
$$
	(M_G f)(x) = (M \circ f \circ M^{-1})(x).
$$
Obviously the inverse transformation $ M^{-1}_G : G_2 \lra G_1 $ is given by 
$$
	(M^{-1}_G f)(x) = (M^{-1} \circ f \circ M)(x) 
$$
for $ f \in G_2 $.

We just have to show that this is a group homomorphism.  

For $ f_1, f_2 $ in $ G_1 $
$$ M_G(f_1 \circ f_2) = M\circ f_1 \circ f_2 \circ M^{-1} = M\circ f_1 \circ M^{-1} \circ M \circ f_2 \circ M^{-1} = M_G(f_1) \circ M_G(f_2) $$
and so $ M_G $ is a group homomorphism. 

Thus the groups $ G_1 $ and $ G_2 $ are isomorphic. 

\section{}
Let $ D $ be a multiply connected domain. Let $ D_0 $ be its convex hull. Then pick an $ a \in D_0 \setminus D $. Such an $ a $ must exist as $ D $ is multiply connected.

Then the function $ f : D \lra \Cplx $, $$ f(z) = \frac{1}{z-a} $$ cannot be analytically extended to all of $ D_0 $. 


\section{}

We can identify any Mobius transformation 
\begin{align}
    f(z) = \frac{az + b}{cz + d} \sim \left[ \begin{array}{cc} a & b \\ c & d \end{array}\right]
\end{align}
with a matrix in $ GL_2(\mathbb{C}) $.

Also see that if
\begin{align}
    f \sim \left[ \begin{array}{cc} a & b \\ c & d \end{array}\right] \\
    g \sim \left[ \begin{array}{cc} e & f \\ g & h \end{array}\right] 
\end{align}
then
\begin{align}
    (g \circ f) &\sim \left[ \begin{array}{cc} e & f \\ g & h \end{array}\right] \left[ \begin{array}{cc} a & b \\ c & d \end{array} \right] \\
    &\sim \left[ \begin{array}{cc} ea + fc & eb + fd \\ ag + ch & gb + dh\end{array}\right].
\end{align}
Noice that this identificaiton is not unqiue as
\begin{align}
    \frac{az + b}{cz+d} \sim \lambda\left[ \begin{array}{cc} a & b \\ c & d \end{array}\right]
\end{align}
for $ \lambda \in \Cplx $.
It turns out that the group of automorphisms on $ \hat{\Cplx} $ (the projective line) is isomorphic the the projective linear group, that is
\begin{align}
    \operatorname{Aut}(\hat{\Cplx}) \cong \operatorname{PGL}(2,\Cplx).
\end{align}
We are interested finite subgroups of $ \operatorname{Aut}(\hat{\Cplx}) $ and hence we are interested in finite subgroups of $ \operatorname{PGL}(2, \Cplx) $.



\end{document}
