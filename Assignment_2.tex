\documentclass{unswmaths}
\usepackage{unswshortcuts}
\usepackage{dsfont}
\begin{document}
\author{Adam J. Gray}
\title{Assignment 1}
\subject{Complex Analysis}
\studentno{3329798}

\newcommand{\llra}{\Leftrightarrow}

\unswtitle

\section{}
\section{}
\section{}

Let $ G_1 $ be the group of all conformal mappings from the unit ball into itself and let $ G_2 $ be the set of all conformal mappings from the upper half plane into itself. The group operation in both cases is function composition.  We claim that these two groups are isomorphic. 

Let $ D_1 = \{ z \in \Cplx : |z| < 1 \} $ and let $ U = \{ z : \Im(z) > 1 \} $.

Define the mapping $ M : D_1 \lra U $ by
$$ M(z) = \frac{(z+1)(1-i)}{2(z-i)}.$$
This is a Mobius transformation, so it is a conformal mapping and importantly bijective. 

We can produce an \emph{induced} transformation $ M_G : G_1 \lra G_2 $ where for $ f \in G_1 $
$$
	(M_G f)(x) = (M \circ f \circ M^{-1})(x).
$$
Obviously the inverse transformation $ M^{-1}_G : G_2 \lra G_1 $ is given by 
$$
	(M^{-1}_G f)(x) = (M^{-1} \circ f \circ M)(x) 
$$
for $ f \in G_2 $.

We just have to show that this is a group homomorphism.  

For $ f_1, f_2 $ in $ G_1 $
$$ M_G(f_1 \circ f_2) = M\circ f_1 \circ f_2 \circ M^{-1} = M\circ f_1 \circ M^{-1} \circ M \circ f_2 \circ M^{-1} = M_G(f_1) \circ M_G(f_2) $$
and so $ M_G $ is a group homomorphism. 

Thus the groups $ G_1 $ and $ G_2 $ are isomorphic. 

\section{}
Let $ D $ be a multiply connected domain. Let $ D_0 $ be its convex hull. Then pick an $ a \in \partial D $. Such an $ a $ must exist in $ D_0 $ as $ D $ is multiply connected.

Then the function $ f : D \lra \Cplx $, $$ f(z) = \frac{1}{z-a} $$ cannot be analytically extended to all of $ D_0 $. The reason being that no only an immediate continuation could be applied as it is in the boundary and clearly no immediate continuation would exist.


\section{}

We can identify any Mobius transformation 
\begin{align}
    f(z) = \frac{az + b}{cz + d} \sim \left[ \begin{array}{cc} a & b \\ c & d \end{array}\right]
\end{align}
with a matrix in $ GL_2(\mathbb{C}) $.

Also see that if
\begin{align}
    f \sim \left[ \begin{array}{cc} a & b \\ c & d \end{array}\right] \\
    g \sim \left[ \begin{array}{cc} e & f \\ g & h \end{array}\right] 
\end{align}
then
\begin{align}
    (g \circ f) &\sim \left[ \begin{array}{cc} e & f \\ g & h \end{array}\right] \left[ \begin{array}{cc} a & b \\ c & d \end{array} \right] \\
    &\sim \left[ \begin{array}{cc} ea + fc & eb + fd \\ ag + ch & gb + dh\end{array}\right].
\end{align}
Notice that this identification is not unique as
\begin{align}
    \frac{az + b}{cz+d} \sim \lambda\left[ \begin{array}{cc} a & b \\ c & d \end{array}\right]
\end{align}
for $ \lambda \in \Cplx $.

Either way we are interested in Mobius transformations which map the unit disk $ D_1 $ to itself. We claim that this consists of all the Mobius transformations of the form
\begin{align}
	f(z) = e^{i\theta} \frac{z+b}{\bar{b} z + 1 } \sim e^{i\theta} \left[ \begin{array}{cc} 1 & b \\ \bar{b} & 1\end{array} \right]
\end{align}
where $ \theta \in \Rl $ and $ b \in \Cplx $ with $ |b| < 1 $. 

This is possible to see from the three point construction of a Mobius transformation. This guarantees that the transformation specified by three points is unique. 

We now want to break this generic transformation up into its finite subgroups.

Suppose firstly that $ \theta = [0] $. Then the fixed points of the transformation are found by
\begin{align}
	z 	&= \frac{z + b}{\bar{b}z + 1} \\
	z(\bar{b}z + 1) &= z + b \\
	\bar{b}z^2 &= b \\
	z = \sqrt{\frac{b}{\bar{b}}}.
\end{align}

So the group of transformations generated (through repeated application) by
\begin{align}
	\frac{z+b}{\bar{b}{z} + 1}
\end{align}
has a global fixed point (acutally 2) given by $ z = \sqrt{ z / \bar{z}} $.

If we allow $ \theta \neq [0] $ then the fixed points correspond to the solutions of the quadratic
\begin{align}
	z(\bar{z} + 1) = e^{i\theta}(z+ b).
\end{align}
We require the group to be finite. This corresponds to 
requiring that $ \theta \in 2\pi \mathbb{Q} $ and to
\begin{align}
	\operatorname{det} \left[ \begin{array}{cc} 1 & b \\ \bar{b} & 1 \end{array} \right] = 1
\end{align} 
but this last requirement means that $ b = 0 $ which just leaves us with transformations of the form
$ f(z) = e^{i \theta} z $ which are just rotations. So long as $ \theta \in 2\pi\mathbb{Q} $ the group generated by any number of rotations will be finite, with a singled global fixed point of 0.

\end{document}
